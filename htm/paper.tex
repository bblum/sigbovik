%-----------------------------------------------------------------------------
%
%               Template for sigplanconf LaTeX Class
%
% Name:         sigplanconf-template.tex
%
% Purpose:      A template for sigplanconf.cls, which is a LaTeX 2e class
%               file for SIGPLAN conference proceedings.
%
% Guide:        Refer to "Author's Guide to the ACM SIGPLAN Class,"
%               sigplanconf-guide.pdf
%
% Author:       Paul C. Anagnostopoulos
%               Windfall Software
%               978 371-2316
%               paul@windfall.com
%
% Created:      15 February 2005
%
%-----------------------------------------------------------------------------


%\documentclass[preprint]{sigplanconf}
\documentclass[10pt]{sigplanconf}

% The following \documentclass options may be useful:
%
% 10pt          To set in 10-point type instead of 9-point.
% 11pt          To set in 11-point type instead of 9-point.
% authoryear    To obtain author/year citation style instead of numeric.

\usepackage{yfonts}
\usepackage{amsmath}
\usepackage{amsthm}
\usepackage{amssymb}
%\usepackage{mathpartir}
\usepackage{hyperref}
\usepackage{url}
\usepackage{graphics}
\usepackage{graphicx}
\usepackage{wasysym}
\usepackage{harmony}
\usepackage{marvosym}
\usepackage{multirow}
\usepackage[usenames,dvipsnames]{xcolor}
\usepackage[utopia]{mathdesign}
\usepackage{natbib}
\usepackage[mathcal]{euscript}
\usepackage[linesnumbered,ruled]{algorithm2e}

\renewcommand{\UrlBreaks}{\do\/\do\a\do\b\do\c\do\d\do\e\do\f\do\g\do\h\do\i\do\j\do\k\do\l\do\m\do\n\do\o\do\p\do\q\do\r\do\s\do\t\do\u\do\v\do\w\do\x\do\y\do\z}

% ____________________________________________________________
% Listings Package Configuration
% \usepackage[scaled]{beramono}

%\renewcommand*\ttdefault{txtt}
\usepackage[T1]{fontenc}

% This Deep Tex Voodoo is from
%   <http://www.latex-community.org/forum/viewtopic.php?f=5&t=2072>
% It's purpose is to make \lstinline normal size, without affecting
% \lstinputlisting.  It seems to work but I have no idea how or why,
% and I rather hope never to learn.
%\makeatletter
%\lst@AddToHook{TextStyle}{\let\lst@basicstyle\ttfamily\normalsize}
%\makeatother

\begin{document}

\conferenceinfo{SIGBOVIK '17}{Pittsburgh, PA, USA}
\copyrightyear{2017}
\copyrightdata{}

\titlebanner{banner above paper title}        % These are ignored unless
\preprintfooter{short description of paper}   % 'preprint' option specified.

\title{
Transactional Memory Concurrency Verification with Landslide
%That `Boring' Stuff Was Part of the Title, BTW. \\
%So was that. And that, and this too. \\
%You got it all, right? \\
%Or Just, ``More Boring Crap about ITG'', for Short. \\
%Oh, That Was Also Part of the Title.
}
% \subtitle{\em The Randomly-Scoped Lambda Calculus}
% \subtitle{Subtitle Text, if any}

\authorinfo{Ben Blum}{}{bblum@cs.cmu.edu}

\maketitle

\begin{abstract}
Hardware transactional memory is a recently-introduced concurrent programming paradigm
%backed by a set of new instructions on recent x86 CPUs,
which allows programmers to elide locks for performance in low-contention workloads.
However,
%using this feature
it
comes at a cost in implementation complexity:
%programmers must support their
fast-path code must be accompanied by backup paths to handle transaction failure.

We extend Landslide, a popular stateless model checker,
with a concurrency model for transactional memory
and evaluate it on several real-world transactional
benchmarks and data structure implementations.
\end{abstract}

\category{D.1.3}{Programming Techniques}{Concurrent Programming}
\category{D.2.4}{Software Engineering}{Software/Program Verification}

\keywords
landslide terminal, baggage claim, ground transportation, ticketing


\section{Introduction}
Transactional Synchronization Extensions (TSX) \cite{transactional-memory}
is an instruction set extension for x86 CPUs which adds hardware-based transactional memory.
The processor uses its existing cache coherence algorithm to
check for memory conflicts with other threads while temporarily staging a sequence of memory accesses.
If no other CPU accesses the same memory during the transaction,
the access sequence is committed to main memory atomically (with respect to visibility to other CPUs).
Otherwise, the operations are discarded and the transaction returns a failure code.

This feature can be used to replace conventional locking in performance-critical concurrent programs.
When a concurrent workload largely accesses thread-local data,
or disjoint sections of a shared data structure,
the contention rate between threads is low,
and transactions will often succeed.
Compared to programs which use conventional locks,
which use bus-locking atomic accesses even in the fastest code path,
TSX provides substantial performance improvements in such programs \cite{htm-experience, htm-performance, htm-mario}.
However, the possibility of transaction failure introduces additional implementation complexity:
programmers must also provide a backup plan
to safely resolve contention between threads,
usually involving conventional synchronization.
These backup paths must coordinate not only with other backup paths
but also with other fast paths which another thread may begin after the original transaction failed,
which even in the simplest transactions requires complex synchronization sequences
\cite{htm-mario}.
This introduces an additional dimension of nondeterminism into an already concurrent program,
and moreover, because transaction failure is expected to be rare,
% だからこそ
obscure interleavings between failure paths are difficult to expose during testing.

This motivates the use of stateless model checking (MC) \cite{verisoft}
to comprehensively verify these transactional programs,
fast paths failure paths and all.
MC aims to force the system to execute all possible thread interleavings under a given test case,
exhaustively checking for bugs or verifying their absence in the corresponding state space.
Many such model checkers exist, of varying flavours of interleaving granularity, memory analysis, and types of programs checked
\cite{chess,dbug-ssv,spin,inspect,r4}.
This work builds upon Landslide \cite{landslide,quicksand},
a simulator-based checker which checks both user- and kernel-level programs at the granularity of individual memory accesses.
Our contributions are as follows:

\begin{enumerate}
	\item We extend Landslide's concurrency model to include transaction failure as an additional source of nondeterminism;
	\item We provide a proof sketch that our implementation matches HTM's execution semantics,
	\item We evaluate the extended Landslide on several transactional programs, analyzing both its bug-finding and verification performance.
\end{enumerate}

The paper is organized as follows.
Section \ref{sec:intro} introduces the problem domain and motivates our research.
The rest of the sections state the rest of the paper.

%%%%%%%%%%%%%%%%%%%%%%%%%%%%%%%%%%%%%%%%%%%%%%%%%%%%%%%%%%%%%%%%%%%%%%%%%%%%%%%%

\section{Background}

TSX was supported on consumer hardware for the first time on Intel's Haswell architecture \cite{htm-haswell},
which provides the {\tt xbegin}, {\tt xend}, and {\tt xabort} instructions for beginning, committing, and aborting
transactions, respectively.

Under software transactional memory (STM) \cite{stm-pldi06},
memory conflicts with other threads and explicit aborts by the programmer are the only reason for transaction failure;
hence, depending on program semantics, some transactions may be guaranteed to succeed.
However, hardware implementations (HTM) may also fail transactions
for several other reasons such as random system interrupts or exhausting the CPU's cache capacity.
Because timer interrupts can in principle occur at any moment,
and with arbitrary frequency (observable by the program, perhaps as a result of a heavily-loaded system),
in this paper we will simplify the failure model by saying that HTM transactions can fail for any reason.


%%%%%%%%%%%%%%%%%%%%%%%%%%%%%%%%%%%%%%%%%%%%%%%%%%%%%%%%%%%%%%%%%%%%%%%%%%%%%%%%

\section{Conclusion}

Please accept my paper.
I worked hard on it.

%%%%%%%%%%%%%%%%%%%%%%%%%%%%%%%%%%%%%%%%%%%%%%%%%%%%%%%%%%%%%%%%%%%%%%%%%%%%%%%%

\bibliographystyle{abbrv}
\bibliography{citations}

\end{document}
