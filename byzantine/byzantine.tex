% v2-acmlarge-sample.tex, dated March 6 2012
% This is a sample file for ACM large trim journals
%
% Compilation using 'acmlarge.cls' - version 1.3, Aptara Inc.
% (c) 2011 Association for Computing Machinery (ACM)
%
% Questions/Suggestions/Feedback should be addressed to => "acmtexsupport@aptaracorp.com".
% Users can also go through the FAQs available on the journal's submission webpage.
%
% Steps to compile: latex, bibtex, latex latex
%
\documentclass[prodmode,acmtap]{acmlarge}

% Metadata Information
\acmVolume{4}
\acmNumber{3}
\acmArticle{} % value doesn't matter
\articleSeq{34} % minimum value to make the right margin black box disappear
\acmYear{1982}
\acmMonth{7}

% Package to generate and customize Algorithm as per ACM style
\usepackage[ruled]{algorithm2e}
\SetAlFnt{\algofont}
\SetAlCapFnt{\algofont}
\SetAlCapNameFnt{\algofont}
\SetAlCapHSkip{0pt}
\IncMargin{-\parindent}
\renewcommand{\algorithmcfname}{ALGORITHM}

\newcommand\prob{Problem of Heads of a Fighting Force from Long Ago}
% Page heads
\markboth{L. Lamport, R. Shostak, and M. Pease}{The \prob}

% Title portion
\title{The \prob}
\author{LESLIE LAMPORT, ROBERT SHOSTAK, and MARSHALL PEASE \\
SRI In Many Lands
}
% NOTE! Affiliations placed here should be for the institution where the
%       BULK of the research was done. If the author has gone to a new
%       institution, before publication, the (above) affiliation should NOT be changed.
%       The authors 'current' address may be given in the "Author's addresses:" block (below).
%       So for example, Mr. Fogarty, the bulk of the research was done at UIUC, and he is
%       currently affiliated with NASA.

\begin{abstract}
Sometimes, in a group of computers, there will be a broken part that will tell
confusing facts to the rest of the computers in the group. If we want to be
able to trust the entire group, it must be able to ignore these confusing facts
when deciding what to do. We can talk about this situation by playing
make-believe about the heads of a fighting force from long ago, who are waiting
with their people around a bad guy city. The people must agree upon a shared
fighting plan, but can only talk to each other by sending a person who carries
a letter. However, one or more of the heads may be bad guys who will try to
confuse the others. The problem is to find a way for the good guys to talk to
each other, without knowing who the bad guys are, to make sure they can agree
on what to do anyway. We will show that, if the good guys use only
word-of-mouth, this problem is possible if and only if more than two-thirds of
the heads are good guys. This means that a single bad guy could confuse two
good guys. If the good guys use signed written letters, which means a bad guy
couldn't pretend their letter came from someone else, the problem is possible
for any number of heads and possible bad guys.
Finally we will talk about ways to use the plans that are answers to this
problem in different situations in the real world.
\end{abstract}

% Words about what this paper is about:
\category{C.2.4.}{Computer-Talking Groups}{Groups of many computers}[groups of brains of computers]
\category{D.4.4}{Computer Brains}{Managing Talking}[talking in groups]
\category{D.4.5}{Computer Brains}{Being Trusted}[accepting faults]

%Big Picture Words
\terms{Plans, Being Trusted}

%More Key Words and Groups of Words
\keywords{Friends agreeing with themselves}

%\acmformat{Daniel Pineo, Colin Ware, and Sean Fogarty. 2009. Neural Modeling of Flow Rendering Effectiveness.}
% At a minimum you need to supply the author names, year and a title.
% IMPORTANT:
% Full first names whenever they are known, surname last, followed by a period.
% In the case of two authors, 'and' is placed between them.
% In the case of three or more authors, the serial comma is used, that is, all author names
% except the last one but including the penultimate author's name are followed by a comma,
% and then 'and' is placed before the final author's name.
% If only first and middle initials are known, then each initial
% is followed by a period and they are separated by a space.
% The remaining information (journal title, volume, article number, date, etc.) is 'auto-generated'.

\begin{document}
\setcounter{page}{382}

\begin{bottomstuff}
This work was helped in part by the People Who Go To Space under plan NAS1-15428 Change 3, the Group for Being Safe Against Flying Fire Guns under plan DASG60-78-C-0046, and the Fighting Force Work Office under plan DAAG29-79-C-0102.

Where the people who wrote this live: Computer Study Work Place, SRI In Many Parts of the World, 333 Flying Animal's Wood Street, Park with a Person's Name, CA 94025.
\end{bottomstuff}


\maketitle

% Head 1
\section{Opening Words}

A group of computers that can be trusted must be able to live with the breaking of one or more of its parts.
A broken part may act in a way that people often don't think about -- that is, sending facts that don't agree to different parts of the group.
The problem of living with this type of breaking is talked about with make-believe as the \prob. We spend the biggest part of the paper to talking about this make-believe problem and finish by showing how our answers can be used in building a group of computers that can be trusted.

We imagine that several parts of the fighting force from long ago are waiting outside a bad guy city, each group controlled by its own head.
The heads can talk with one another only by a person who carries a letter.
After watching the bad guys, they must decide upon a shared plan of attack.
However, some of the heads may be bad guys, trying to stop the good guys heads from agreeing. The heads must have an plan to make sure that \\
\\
A. All good guys decide upon the same plan of how to act. \\

The good guy heads will all do what the plan says they should, but the bad guy heads may do anything they wish. The plan must make sure of situation A no matter what the bad guys do.

\newcommand\fact{\ensuremath{\mathsf{fact}}}

The good guys should not only agree, but should agree upon a plan that makes sense. So, we also want to make sure that \\
\\
B. A small number of bad guy heads can't cause the good guy heads to take a bad plan. \\

Situation B is hard to make clear, since it needs you to say exactly what a bad plan is, and we do not attempt to do so. Instead, we consider how the heads reach a way to agree. Each head watches the bad guy city and tells what he or she sees to the others. Let $\fact(i)$ be the facts told by the head with number $i$. Each head uses some way to put together the facts $\fact(1) \dots \fact(n)$ into a single plan of what to do, where $n$ is the number of heads. We can make Situation A happen by having all heads use the same way for putting together the facts, and make Situation B happen by using a way that can be trusted. Think about this case: if the only thing to agree on is whether to attack or run away, then $\fact(i)$ can be Head $i$'s thought of which way is best, and the real plan can come from which way has more heads that want to do it. A small number of bad guys can change the way only if there were almost the same number of good guys who wanted to do each way, in which case not either way could be called bad.

While this approach may not be the only way to make situations A and B happen, it is the only one we know of.
It needs a way for the heads to tell their facts $\fact(i)$ to one another. The most clear way is for the head number $i$ to send $\fact(i)$ by a person who carries a letter to each other head. However, this does not work, because making situation A happen needs every good guy head to read the same facts $\fact(1) \dots \fact(n)$, and a bad guy head may send different facts to different heads. For situation A to happen, the following must be true: \\
\\
1. Every good guy head must read the same facts $\fact(1) \dots \fact(n)$. \\

Situation 1 means that a head can't in all cases use a fact of $\fact(i)$ read straight from the head number $i$, since a bad guy head number $i$ may send different facts to different heads.
This means that without care, in meeting situation 1 we might make it possible for the heads to use a fact for $\fact(i)$ different from the one sent by head number $i$--even though that head is good.
We must not allow this to happen if situation B is to be met. Think about this case: we can't allow a few bad guys to cause the good guys to act as if the facts were ``run away'' , \dots, ``run away'' if every good guy said ``attack''.
So, we need the following thing for each $i$: \\
\\
2. If head number $i$ is good, then the fact that he or she sends must be used by every good guy head as the fact for $\fact(i)$. \\

We can say situation 1 another way by saying that for every $i$ (whether or not the head number $i$ is a good guy), \\
\\
1'. Any two good guy heads use the same meaning of $\fact(i)$. \\

Situations 1' and 2 both use just the single fact that was sent by head number $i$.
Because of this, we can think about the smaller problem of how a single head sends their fact to the others. We'll talk about this by talking about a controlling head sending a letter to her friends, causing the following problem. \\

{\em The \prob.} A controlling head must send an order to her n-1 friends such that \\
FA1. All good-guy friends listen to the same order. \\
FA2. If the controlling head is a good guy, then every good guy friend listens to the order she sends. \\

These situations, FA1 and FA2, are called the {\em friends agreeing} situations. Note that if the controlling head is a good guy, then FA1 follows from FA2. However, the controlling head need not be good.

To fix our first problem, head number i sends their fact of $\fact(i)$ by using an answer to the Fighting Force Heads Problem to send the order ``use $\fact(i)$ as my fact'', with the other heads acting as the helping friends.

\section{Showing What Isn't Possible}

The Problem of Heads of a Fighting Force from Long Ago seems simple, but it is actually not. It is hard because of the surprising fact that if the heads can send only talk to each other out loud, then any plan can only work if more than two-thirds of the heads are good guys. Now, when we say ``talking out loud'', we mean that the words are completely under the control of the person saying them, so a bad guy could say anything he or she wanted to, even ``Your good guy friend said to do such-and-such!''
This sort of speaking is the same as how computers usually speak to one another. In Part 4, we consider signed, written letters, for which this is not true.

We now show that with spoken words, no plan for three people can handle a single bad guy.
To keep things simple, we consider the case in which the only things to do are ``attack'' or ``run away''.
Let us first think about the situation pictured in Figure 1 in which the first head is a good guy, and sends an "attack" order, but Friend 2 is a bad guy and tells Friend 1 that he heard a ``run away'' order. For FA2 to happen, Friend 1 must listen to the order to attack.

Now consider another situation, shown in Figure 2, in which the first head is actually a bad guy, and sends an ``attack'' order to Friend 1 and a ``run away'' order to Friend 2. Friend 1 does not know who the bad guy is, and she can't tell what the first head actually said to Friend 2. So, the situations in these two pictures appear exactly the same to Friend 1. If the bad guy lies all the time, then there is no way for Friend 1 to know which situation is happening, so she must listen to the ``attack'' order in both of them. So, any time Friend 1 hears an ``attack'' order from the first head, she must listen to it.

However, in the same way we could show that if Friend 2 hears a ``run away'' order from the first head then he must listen to it even if Friend 1 tells him that the head said "attack". Because of this, in Figure 2, Friend 2 must listen to the ``run away'' order while Friend 1 listens to the ``attack'' order, which means situation FA1 doesn't happen. So, there is no way for three heads to agree if one of them is a bad guy.

This way of thinking about it may appear right, but you should think hard before believing such hand-waving reasoning. Although what we said is right after all, we have seen other ``ways of thinking about it'' that are totally wrong.
We know of no area in the study of computers or the study of numbers in which hand-waving reasoning could more easily lead to wrong answers than in the study of this type of problem. For a stronger way of thinking about why this isn't possible, you should read [3]. % TODO cite
Using this answer, we can show that no plan can work for a given number of heads if one third of them are bad guys.
% Omitted: Albanian generals contradiction proof.

You might think that it's so hard to fix the \prob because the heads need to agree with each other completely. Actually, this is not the case, and only-sort-of agreeing with each other is just as hard. Suppose that instead of trying to agree on a complete plan to attack, the heads must agree only upon a set of times, during which the attack should happen. We will say the first head needs to order the time of the attack, and the following two situations need to happen: \\
\\
FA1'. All good guys attack within 10 minutes of one another. \\
FA2'. If the first head is a good guy, then every other good guy attacks within 10 minutes of the time given in the first head's order. \\
\\
(We are supposing that the orders are given and thought about the day before the attack and that the time at which an order is heard doesn't matter--only the attack time given in the order matters.)



% Bibliography
\bibliographystyle{ACM-Reference-Format-Journals}
\bibliography{citations}

% History dates
\received{Month Four 1980}{Month Ten-and-One 1981}{Month Four 2013}


% \elecappendix
% \section{Analysis of Invalid Trials}

\end{document}
