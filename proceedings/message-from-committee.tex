\documentclass[12pt]{article}
\usepackage{graphicx}
\usepackage{multirow}
\usepackage{fullpage}
\usepackage{times}
\usepackage[normalem]{ulem}
\setlength\parindent{0pt}
\setlength\parskip{12pt}

\begin{document}

{\sffamily
\begin{tabular}{ll}
\multirow{3}{*}{\includegraphics[width=1in]{ach.png}}\\
& \textbf{\Huge{SIGBOVIK 2014}} \\ &\\
& \LARGE{Message from the Organizing Committee} \\
&\\
\hline
\end{tabular}}
\vspace{2em}

{\large \bf \sffamily Multiple-Choice Section}

\begin{enumerate}
\item {\em 5 points.} Please indicate below the emotions with which the SIGBOVIK Organizing Committee presents this document. Circle up to three, but no fewer than four, choices.
\begin{enumerate}
\item Joy
\item Embarrassment
\item Surprise
\item Admiration
\item Arrogance
\item Pride
\item Terror
\end{enumerate}
\item {\em 5 points.} Which is the correct full name of this conference?
\begin{enumerate}
\item The Eighth Annual Intercalary Conference about Symposium on Human Dance Party of Workshop in Celebration of Harry W. Bovik's $2^6$th Birthday.
\item The Eightieth Annual Intracalary Workshop about Workshop on Robot Dance Party of Conference in Mourning of Harry Q. Bovik's $2^6$th Birthday.
\item The Eighth Centennial Intercalary Workshop about Symposium on Robot Karaoke Party of Conference in Celebration of Harry Q. Bovik's $3^6$th Birthday.
\item The Eighth Annual Intercalary Workshop about Symposium on Robot Dance Party of Conference in Celebration of Harry Q. Bovik's $2^6$th Birthday.
\end{enumerate}

\end{enumerate}

{\large \bf \sffamily Reading Comprehension Section}

{\em 25 points.}

This year's SIGBOVIK marks our eighth continuous year of high-quality research in a rich variety of topics, including but not limited to applied phlogistonics, synergistic hyperparadigmatism, and elbow macaroni. Eight is a very special annuality of this conference, as it not only breaks the ``highest annuality'' record set last year, but also is the fourth power of two number of years for which this conference has been in existence. If you're not impressed yet, note that the fourth power of two is a very special power of two, as four is the second power of two, and you know what they say about the second power of two.

As it is a very special year, we have decided to conduct a survey of past SIGBOVIKs, or SIGBOVIX for short, classifying the various papers to get a feel for where future SIGBOVIKs might be headed. This was driven by the realization depicted in Figure~\ref{fig:table}.

\begin{figure}[h]
\begin{center}
\begin{tabular}{p{5em}|p{5em}|p{5em}|}
& Serious treatment & Humorous treatment \\
\hline
Serious idea & \bf Mainstream conferences & \bf SIGBOVIK \\
\hline
Humorous idea & \bf SIGBOVIK & \bf SIGBOVIK \\
\hline

\end{tabular}
\end{center}
\caption{A research paper is comprised of an {\em idea} and a {\em treatment} of that idea. Each of the idea and the treatment may be either serious or humorous.}
\label{fig:table}
\end{figure}

Observe that SIGBOVIK offers a venue for three times as many different types of research as ``mainstream'' conferences. We won't belabour the obvious conclusion that SIGBOVIK is superior, although, just saying, you know; rather, we are interested in distinguishing among SIGBOVIK's three major categories of research.
\begin{enumerate}
\item {\bf Humorous idea, humorous treatment.} The most common, but by no means the most lowly, of SIGBOVIK publications. The research contribution is typically contained entirely within the paper itself; no separate artefact is constructed.
\item {\bf Serious idea, humorous treatment.} The rarest specimen of SIGBOVIK research. Often a retelling of famous events, people, or theoretical results from mainstream computer science.
\item {\bf Humorous idea; serious treatment.} Frequently denoted by independent artefacts accompanying the paper submission, such as a website, compiler implementation, hardware, or proof. Such publications would often be suitable for acceptance at ``mainstream'' conferences, were the core idea not out of scope.
\end{enumerate}

The results of our survey are shown in Figure~\ref{fig:results} \sout{below}on the next page.

\begin{figure}[h]
\begin{center}
\begin{tabular}{r|c|c|c|c|c|c|c|c|}
& \multicolumn{8}{c|}{SIGBOVIK year of incidence} \\
& 2007 & 2008 & 2009 & 2010 & 2011 & 2012 & 2013 & 2014 \\
\hline
\hline
Category 1 & 0 & 0 & 0 & 0 & 0 & 0 & 0 & 0 \\
\hline
Category 2 & 0 & 0 & 0 & 0 & 0 & 0 & 0 & 0 \\
\hline
Category 3 & 0 & 0 & 0 & 0 & 0 & 0 & 0 & 0 \\
\hline
Other & 0 & 0 & 0 & 0 & 0 & 0 & 0 & 0 \\
\hline
\hline
Total publications & 0 & 0 & 0 & 0 & 0 & 0 & 0 & 0 \\

\end{tabular}
\end{center}
\caption{Results.}
\label{fig:results}
\end{figure}


\end{document}
