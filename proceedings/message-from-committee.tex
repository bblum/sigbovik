\documentclass[12pt]{article}
\usepackage{graphicx}
\usepackage{multirow}
\usepackage{fullpage}
\usepackage{times}
\usepackage[normalem]{ulem}
\setlength\parindent{0pt}
\setlength\parskip{12pt}

\begin{document}

{\sffamily
\begin{tabular}{ll}
\multirow{3}{*}{\includegraphics[width=1in]{ach.png}}\\
& \textbf{\Huge{SIGBOVIK 2014}} \\ &\\
& \LARGE{Message from the Organizing Committee} \\
&\\
\hline
\end{tabular}}
\vspace{2em}

{\large \bf \sffamily Multiple-Choice Section}
\begin{enumerate}
\item {\em 5 points.} Please indicate below the emotions with which the SIGBOVIK Organizing Committee presents this document. Circle up to three, but no fewer than four, choices.
\begin{enumerate}
\item Joy
\item Embarrassment
\item Surprise
\item Admiration
\item Arrogance
\item Pride
\item Terror
\end{enumerate}
\item {\em 5 points.} Which is the correct full name of this conference?
\begin{enumerate}
\item The Eighth Annual Intercalary Conference about Symposium on Human Dance Party of Workshop in Celebration of Harry W. Bovik's $2^6$th Birthday.
\item The Eightieth Annual Intracalary Workshop about Workshop on Robot Dance Party of Conference in Mourning of Harry Q. Bovik's $2^6$th Birthday.
\item The Eighth Centennial Intercalary Workshop about Symposium on Robot Karaoke Party of Conference in Celebration of Harry Q. Bovik's $3^6$th Birthday.
\item The Eighth Annual Intercalary Workshop about Symposium on Robot Dance Party of Conference in Celebration of Harry Q. Bovik's $2^6$th Birthday.
\end{enumerate}

\end{enumerate}

{\large \bf \sffamily Reading Comprehension Section}

{\em 25 points.}

This year's SIGBOVIK marks our eighth continuous year of high-quality research in a rich variety of topics, including but not limited to applied phlogistonics, synergistic hyperparadigmatism, and elbow macaroni. Eight is a very special annuality of this conference, as it not only breaks the ``highest annuality'' record set last year, but also is the fourth power of two number of years for which this conference has been in existence. If you're not impressed yet, note that the fourth power of two is a very special power of two, as four is the second power of two, and you know what they say about the second power of two.

As it is a very special year, we have decided to conduct a survey of past SIGBOVIKs, or SIGBOVIX for short, classifying the various papers to get a feel for where future SIGBOVIKs might be headed. This was driven by the realization depicted in Figure~\ref{fig:table}.

\begin{figure}[h]
\begin{center}
\begin{tabular}{p{5em}|p{5em}|p{5em}|}
& Serious treatment & Humorous treatment \\
\hline
Serious idea & \bf Mainstream conferences & \bf SIGBOVIK \\
\hline
Humorous idea & \bf SIGBOVIK & \bf SIGBOVIK \\
\hline

\end{tabular}
\end{center}
\caption{A research paper is comprised of an {\em idea} and a {\em treatment} of that idea. Each of the idea and the treatment may be either serious or humorous.}
\label{fig:table}
\end{figure}

Observe that SIGBOVIK offers a venue for three times as many different types of research as ``mainstream'' conferences. We won't belabour the obvious conclusion that SIGBOVIK is superior, although, you know, just getting that out there, we were all thinking it; rather, we are interested in distinguishing among SIGBOVIK's three major categories of research.
\begin{enumerate}
\item {\bf Humorous idea, humorous treatment.} The most common, but by no means the most lowly, of SIGBOVIK publications. The research contribution is typically contained entirely within the paper itself; no separate artefact is constructed.
\item {\bf Serious idea, humorous treatment.} The rarest specimen of SIGBOVIK research. Often a retelling of famous events, people, or theoretical results from mainstream computer science.
\item {\bf Humorous idea; serious treatment.} Frequently denoted by independent artefacts accompanying the paper submission, such as a website, compiler implementation, hardware, or proof. Such publications would often be suitable for acceptance at ``mainstream'' conferences, were the core idea not out of scope.
\end{enumerate}

\sout{I} We surveyed the proceedings of past SIGs BOVIK \sout{the night before finalizing the proceedings and sending them off to Lulu} like responsible researchers.
Our methodology is definitely completely immune to any biases that might be caused by the survey not being blinded in any way or by it being conducted by only one person, and {\em definitely} free from any bias related to the study subject knowing in advance which conclusions would be drawn.
%Moving right along.
The results of our survey are shown in Figures~\ref{fig:results} and~\ref{fig:results2} \sout{below} on the next page.

\begin{figure}[h]
\begin{center}
\begin{tabular}{r|c|c|c|c|c|c|c|c|}
& \multicolumn{8}{c|}{SIGBOVIK year of incidence} \\
& 2007 & 2008 & 2009 & 2010 & 2011 & 2012 & 2013 & 2014 \\
\hline
\hline
Humorous treatment of humorous idea & 32 & 20 & 30 & 33 & 16 & 17 & 15 & 16 \\ \hline
Humorous treatment of serious idea  & 2  & 0  & 1  & 0  & 0  & 1  & 2  & 0 \\ \hline
Serious treatment of humorous idea  & 2  & 4  & 3  & 0  & 4  & 6  & 6  & 8 \\ \hline
Other                               & 1  & 1  & 5  & 3  & 0  & 2  & 2  & 0 \\ \hline
\hline
Total publications                  & 37 & 25 & 39 & 36 & 20 & 26 & 25 & 24 \\

\end{tabular}
\end{center}
\caption{Results. The `other' category comprises submissions that defy my perfect classification scheme, including cryptic iconography, video games, comics, choose your own adventures, and take-home midterm examinations.}
\label{fig:results}
\end{figure}

\begin{figure}[h]
\begin{center}
\begin{tabular}{r|c|c|c|c|c|c|c|c|}
& \multicolumn{8}{c|}{SIGBOVIK year of incidence} \\
& 2007 & 2008 & 2009 & 2010 & 2011 & 2012 & 2013 & 2014 \\
\hline
\hline
Meta-humor about research
& 7  & 2  & 8  & 8  & 8  & 1  & 2  & 2  \\ \hline
Type theory or programming jokes
& 8  & 3  & 1  & 7  & 2  & 1  & 2  & 4  \\ \hline
Puns
& 4  & 4  & 8  & 1  & 0  & 0  & 2  & 3  \\ \hline
Poop or dick jokes
& 1  & 0  & 1  & 1  & 0  & 1  & 1  & 2  \\ \hline
Politics, economics, or patents jokes
& 0  & 0  & 0  & 0  & 1  & 2  & 2  & 1  \\ \hline
Social media jokes
& 0  & 0  & 0  & 2  & 2  & 1  & 0  & 0  \\ \hline
%\hline
%Food jokes
%& 1  & 0  & 2  & 1  & 0  & 3  & 0  & 0  \\ \hline
%NP completeness jokes
%& 1  & 1  & 0  & 0  & 0  & 1  & 0  & 0  \\ \hline
%Graph jokes
%& 1  & 0  & 0  & 1  & 0  & 0  & 0  & 0  \\ \hline
%Pop culture jokes
%& 1  & 1  & 0  & 0  & 0  & 2  & 2  & 0  \\ \hline
%Computational impossibilities
%& 0  & 1  & 4  & 4  & 1  & 1  & 0  & 0  \\ \hline
%Cat jokes
%& 0  & 1  & 2  & 0  & 1  & 1  & 0  & 0  \\ \hline
%Images / visual gags
%& 1  & 1  & 0  & 1  & 0  & 1  & 1  & 0  \\ \hline
%Robot uprisings, apocalypse, and the supernatural
%& 3  & 2  & 0  & 3  & 0  & 2  & 1  & 1  \\ \hline
%Sex jokes
%& 0  & 0  & 0  & 0  & 1  & 0  & 0  & 0  \\ \hline
%Pittsburgh local humour
%& 0  & 0  & 1  & 1  & 0  & 0  & 0  & 3  \\ \hline
%Other
%& 4  & 4  & 2  & 1  & 0  & 0  & 0  & 0  \\ \hline
Other
& 12 & 11 & 11 & 12 & 3  & 11 & 4  & 4  \\ \hline

\end{tabular}
\end{center}
\caption{Breakdown among category-1 publications of subject of humor. Among this new `other' category dwell comestibility theory, NP-completeness, cat pictures, computational impossibilities, the supernatural, and robot uprisings and other apocalypse situations.}
\label{fig:results2}
\end{figure}

A number of trends are evident. First and foremost is the decline over time of the proportion of category-1 submissions and a corresponding increase in submissions accompanied by an actual artefact that must have been put together with blood, sweat, and so forth, rather than just the cushy endeavour of writing down shallow drivel in LaTeX and calling that a ``research paper''. Go us! Just kidding. We value all SIGBOVIK submissions highly, \sout{except for} especially yours. But really.

Second, we note the prominence among category-1 submissions of meta-research papers; that is, papers about writing papers, giving talks, or improving productivity. It is unsurprising that this is a popular subject for obvious reasons. Finally, we note a modest decline in the proportion of programming language papers, which we attribute to SIGBOVIK's audience's interests expanding beyond the core of its type-theoretician founding mothers and fathers, to include such newfangled topics as Twitterers and Bitdollars.

Without further ado, adon't, or adonuts, the proceedings of your SIGBOVIK \#8.



\end{document}
