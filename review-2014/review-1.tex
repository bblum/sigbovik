\documentclass[12pt]{article}
\usepackage{graphicx}
\usepackage{multirow}
\usepackage{fullpage}
\usepackage{times}
\setlength\parindent{0pt}
\setlength\parskip{12pt}

\begin{document}

{\sffamily
\begin{tabular}{ll}
\multirow{2}{*}{\includegraphics[width=1in]{ach.png}}\\
& \textbf{\Huge{SIGBOVIK 2014 Paper Review}} \\ &\\
& \LARGE{Paper 1: New Results in $k/n$ Power-Hours} \\
&\\
\hline
\end{tabular}}
\vspace{2em}

{\large\bf
\begin{tabular}{l}
Robert Marsh, King Under the Mountain \\
Rating: 3 (weak accept) \\
Confidence: 2/4 \\
\end{tabular}}
\vspace{1em}

Man, figure 5 does look really sweet.  like whoa.  you could take that and trim the edges off and use it as a boss in some kind of diagonal-scrolling space shoot-em-up.  no really, that's blowing my mind.  you could, like make an entire space invaders galaga thingy out of these charts.  it would be rad as all get out.

On the other hand, one must consider the broader implications of research of this sort.
In particular, should this conference encourage an attempt at a $k/n$ power hour for any $k$, even a difficult-to-reach one like $\langle 11, 53\rangle$?  
Let us first note that the finer things in life should be savored.
However, there is such a thing as Beer 30 Light, and people are going to attempt such things whether we help them or not, so we might as well ensure they have the tools they need for slightly more responsible drinking.

In sum, this paper's significant contributions to cool fractally things seems to outweigh its potential harm to society.

\end{document}
