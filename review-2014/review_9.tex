\documentclass[12pt]{article}
\usepackage{graphicx}
\usepackage{multirow}
\usepackage{fullpage}
\usepackage{times}
\setlength\parindent{0pt}
\setlength\parskip{12pt}

\begin{document}

{\sffamily
\begin{tabular}{ll}
\multirow{2}{*}{\includegraphics[width=1in]{ach.png}}\\
& \textbf{\Huge{SIGBOVIK 2014 Paper Review}} \\ &\\
& \LARGE{Paper $9$: A Simple Category-Theoretic ...} \\
&\\
\hline
\end{tabular}}

{\large\bf
\begin{tabular}{l}
Reviewed by: Ed Morehouse (Carnegie Mellon University) \\
\end{tabular}}

	``A Simple Category-Theoretic Understanding of Category-Theoretic Diagrams'' by ``Stefan Muller''
	is a work of incomprehensible abstract nonsense,
	and as such, constitutes a valuable contribution to the field of category theory.
	
	The article seems to be about understanding category
	theory through the graphical language of commutative diagrams.
	I say, ``seems to be'', because after gleaning this much from the abstract,
	I decided to test (what I assume to be) the article's central thesis
	by attempting to understand the article itself by just looking at the pictures.
	
	I infer that the article begins by describing the basic properties
	of morphisms in a category under composition and identity
	 -- although it could be talking about something else entirely, it's kind of hard to say.
%	In figure $3$, I discovered the first major flaw of the paper,
%	here ``Muller'' uses applicative-order notation for composition.
%	It has been well-known since the foundational work of Mac Lane that this is an intellectual dead-end.
	The next bit seems like it might be about constructing a category
	whose objects are themselves commutative diagrams and whose morphisms are affine transformations of such%
	\footnote
	{
		note to self: ask ``Muller'' what the paper is really about, then reject it and steal his idea.
	}.
	
	The paper ends by constructing the ``Morehouse-Sierpinski $\omega$-category of commuting squares''
	(let's call it).
	Although ``Muller'' has (for all I know) now solved a long-standing open problem in higher dimensional category theory
	by giving a finitary, constructive presentation of this important category,
	there is still currently no type-theoretic interpretation (that I could think of in 5 minutes),
	nor are current \LaTeX \, diagram-description packages
	adequate for drawing $\omega$-dimensional commutative diagrams.
	I'll just assume that ``Muller'' leaves these issues for future work.
		
\begin{tabular}{l}
{\large\bf Rating:} I give this article the terminal rating in the co-op of the bicategory of journal reviews. \\
{\large\bf Confidence:} What am I supposed to put here? \\
	Why are the instructions not presented in picture-form, like the ones for assembling my bookshelf: \\
\end{tabular}
\vspace{-1em}
\begin{center}
	{\includegraphics[width=2.5in]{instructions.jpg}}
\end{center}
\end{document}
