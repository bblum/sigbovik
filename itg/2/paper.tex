%-----------------------------------------------------------------------------
%
%               Template for sigplanconf LaTeX Class
%
% Name:         sigplanconf-template.tex
%
% Purpose:      A template for sigplanconf.cls, which is a LaTeX 2e class
%               file for SIGPLAN conference proceedings.
%
% Guide:        Refer to "Author's Guide to the ACM SIGPLAN Class,"
%               sigplanconf-guide.pdf
%
% Author:       Paul C. Anagnostopoulos
%               Windfall Software
%               978 371-2316
%               paul@windfall.com
%
% Created:      15 February 2005
%
%-----------------------------------------------------------------------------


%\documentclass[preprint]{sigplanconf}
\documentclass[10pt]{sigplanconf}

% The following \documentclass options may be useful:
%
% 10pt          To set in 10-point type instead of 9-point.
% 11pt          To set in 11-point type instead of 9-point.
% authoryear    To obtain author/year citation style instead of numeric.

\usepackage{yfonts}
\usepackage{amsmath}
\usepackage{amsthm}
\usepackage{amssymb}
%\usepackage{mathpartir}
\usepackage{hyperref}
\usepackage{url}
\usepackage{graphics}
\usepackage{graphicx}
\usepackage{wasysym}
\usepackage{harmony}
\usepackage{marvosym}
\usepackage{multirow}
\usepackage[usenames,dvipsnames]{xcolor}
\usepackage[utopia]{mathdesign}
\usepackage{natbib}

% ____________________________________________________________
% Listings Package Configuration
% \usepackage[scaled]{beramono}

%\renewcommand*\ttdefault{txtt}
\usepackage[T1]{fontenc}

% This Deep Tex Voodoo is from
%   <http://www.latex-community.org/forum/viewtopic.php?f=5&t=2072>
% It's purpose is to make \lstinline normal size, without affecting
% \lstinputlisting.  It seems to work but I have no idea how or why,
% and I rather hope never to learn.
%\makeatletter
%\lst@AddToHook{TextStyle}{\let\lst@basicstyle\ttfamily\normalsize}
%\makeatother

\begin{document}

\conferenceinfo{SIGBOVIK '17}{Pittsburgh, PA, USA}
\copyrightyear{2017}
\copyrightdata{}

\titlebanner{banner above paper title}        % These are ignored unless
\preprintfooter{short description of paper}   % 'preprint' option specified.

\title{
A Boring Follow-Up Paper to \\
``Which ITG Stepcharts are Turniest?'' \\
Titled, ``Which ITG Stepcharts are Crossoveriest and/or Footswitchiest?'' \\
%That `Boring' Stuff Was Part of the Title, BTW. \\
%So was that. And that, and this too. \\
%You got it all, right? \\
%Or Just, ``More Boring Crap about ITG'', for Short. \\
%Oh, That Was Also Part of the Title.
}
% \subtitle{\em The Randomly-Scoped Lambda Calculus}
% \subtitle{Subtitle Text, if any}

\authorinfo{Ben Blum}{}{bblum@cs.cmu.edu}

\maketitle

\begin{abstract}
	%ITG is a popular dance game in which players step on arrows while listening to music. The arrow patterns, indicated by a {\em stepchart}, may range among any level of complexity and difficulty. Among the many factors contributing to a stepchart's difficulty is how much the player must turn from side to side.
	%Other more obvious factors, such as raw speed, have been well studied in prior work.
	%This paper presents an analytic study of this {\em turniness} factor.
	%We study the turniness of many existing stepcharts, and present a novel (but unsurprising) approach to automatically generating maximally (or minimally) turny charts.
	%Among real-world songs, we find stepcharts with overall turniness ranging from 0\% to 81.33\% of the theoretical maximum.


\end{abstract}

\category{D.D.R.}{Exercise and Fitness}{Arcade Dance Games}

\keywords
in, the, groove


\section{Introduction}


%%%%%%%%%%%%%%%%%%%%%%%%%%%%%%%%%%%%%%%%%%%%%%%%%%%%%%%%%%%%%%%%%%%%%%%%%%%%%%%%

\section{Flashback Scene}


\begin{table}[h]
	\begin{center}
	\begin{tabular}{cc|cccc}
		& & \multicolumn{4}{c}{Right foot} \\
		& & $\leftarrow$ & $\downarrow$ & $\uparrow$ & $\rightarrow$ \\
		\hline
		\multirow{4}{*}{Left foot}
		%                 L   D   U   R
		& $\leftarrow$  & - & $UR$ & $UL$ & $U$ \\
		& $\downarrow$  & $DL$* & - & $L$ & $UL$ \\
		& $\uparrow$    & $DR$* & $R$ & - & $UR$ \\
		& $\rightarrow$ & $D\dagger$ & $DR$* & $DL$* & - \\

	\end{tabular}
	\end{center}
	\caption{Facing directions. ``Crossover'' facings are marked (*), and the ``lateral'' facing is marked ($\dagger$). Note the appealing diagonal symmetry.}
	\label{tab:facing}
\end{table}

% dont forget
% http://dancedancerevolutionddr.wikia.com/wiki/Afronova_walk

%%%%%%%%%%%%%%%%%%%%%%%%%%%%%%%%%%%%%%%%%%%%%%%%%%%%%%%%%%%%%%%%%%%%%%%%%%%%%%%%

\newcommand\hilight[2]{\color{#1}#2\color{black}}
\definecolor{pink}{RGB}{128,0,192}
\definecolor{orange}{RGB}{192,96,0}
\definecolor{olivegreen}{RGB}{0,127,32}
\definecolor{brickred}{RGB}{192,0,0}
\definecolor{commentblue}{RGB}{0,128,192}

\begin{figure*}[t]
\begin{center}
\begin{tabular}{l}
	\texttt{\hilight{orange}{data}~\hilight{olivegreen}{Step}~= \hilight{brickred}{L}~| \hilight{brickred}{D}~| \hilight{brickred}{U}~| \hilight{brickred}{R}~| \hilight{brickred}{Jump}~\hilight{orange}{deriving}~\hilight{olivegreen}{Eq}} \\
\texttt{} \\
\texttt{\hilight{orange}{data}~\hilight{olivegreen}{AnalysisState}~= \hilight{brickred}{S}~\{ steps :: \hilight{olivegreen}{Int}, xovers :: \hilight{olivegreen}{Int}, switches :: \hilight{olivegreen}{Int}, jacks :: \hilight{olivegreen}{Int},} \\
\texttt{~~~~~~~~~~~~~~~~~~~~~~~~ lastStep :: \hilight{olivegreen}{Maybe}~\hilight{olivegreen}{Step}, doubleStep :: \hilight{olivegreen}{Bool}, lastFlip :: \hilight{olivegreen}{Bool},} \\
\texttt{~~~~~~~~~~~~~~~~~~~~~~~~ lastFoot :: \hilight{olivegreen}{Bool}, stepsL :: [\hilight{olivegreen}{Bool}], stepsR :: [\hilight{olivegreen}{Bool}] \}} \\
\texttt{} \\
\texttt{\hilight{pink}{commitStream}~:: \hilight{olivegreen}{AnalysisState}~-> \hilight{olivegreen}{AnalysisState}} \\
\texttt{\hilight{pink}{commitStream}~s = s \{ xovers~~ = xovers~~ s + \hilight{orange}{if}~f \hilight{orange}{then}~ns - nx \hilight{orange}{else}~nx, } \\
\texttt{~~~~~~~~~~~~~~~~~~~~ switches = switches s + \hilight{orange}{fromEnum}~(f == lastFlip s \&\& doubleStep s), } \\
\texttt{~~~~~~~~~~~~~~~~~~~~ jacks~~~~= jacks~~~~s + \hilight{orange}{fromEnum}~(f /= lastFlip s \&\& doubleStep s), } \\
	\texttt{~~~~~~~~~~~~~~~~~~~~ lastFlip = f, stepsL = \hilight{brickred}{[]}, stepsR = \hilight{brickred}{[]}~\}} \\
\texttt{~~~~\hilight{orange}{where}~ns = \hilight{orange}{length}~\$~~~~~~~~~~~~~~stepsL s ++ stepsR s} \\
\texttt{~~~~~~~~~~nx = \hilight{orange}{length}~\$ \hilight{orange}{filter not}~\$ stepsL s ++ stepsR s} \\
\texttt{~~~~~~~~~~\hilight{commentblue}{-{}- if more than half the L/R steps in this stream were crossed over,}} \\
\texttt{~~~~~~~~~~\hilight{commentblue}{-{}- then we got the footing backwards and need to flip the stream.}} \\
\texttt{~~~~~~~~~~\hilight{commentblue}{-{}- as a tiebreaker, flip if the stream is already more jacky than}} \\
\texttt{~~~~~~~~~~\hilight{commentblue}{-{}- footswitchy, i.e., if past streams flipped more often than not.}} \\
\texttt{~~~~~~~~~~f = nx * 2 > ns || nx * 2 == ns \&\& ((jacks s > switches s) /= lastFlip s)} \\
\texttt{} \\
\texttt{\hilight{pink}{analyzeStep}~:: \hilight{olivegreen}{AnalysisState}~-> \hilight{olivegreen}{Step}~-> \hilight{olivegreen}{AnalysisState}} \\
\texttt{\hilight{pink}{analyzeStep}~s step} \\
\texttt{~~~~\hilight{commentblue}{-{}- a jump resets the footing, so the next step can be stepped with either}} \\
\texttt{~~~~\hilight{commentblue}{-{}- foot. commit the stream so far to treat it separately from what follows.}} \\
\texttt{~~~~\hilight{commentblue}{-{}- bracket-jumps are, of course, future work.}} \\
\texttt{~~~~| step == \hilight{brickred}{Jump}~= (\hilight{pink}{commitStream}~s) \{ lastStep = \hilight{brickred}{Nothing}, doubleStep = \hilight{brickred}{False}~\}} \\
\texttt{~~~~\hilight{commentblue}{-{}- two steps on the same arrow might be a jack, or might be a footswitch.}} \\
\texttt{~~~~\hilight{commentblue}{-{}- to figure out which, commit the stream so far, and begin a new stream}} \\
\texttt{~~~~\hilight{commentblue}{-{}- whose footing will retroactively determine how to foot this step.}} \\
\texttt{~~~~\hilight{commentblue}{-{}- also, unlike jumps, this step gets counted as part of the next stream.}} \\
\texttt{~~~~| lastStep s == \hilight{brickred}{Just}~step = stream (\hilight{pink}{commitStream}~s) \{ doubleStep = \hilight{brickred}{True}~\}} \\
\texttt{~~~~\hilight{commentblue}{-{}- a normal streamy step.}} \\
	\texttt{~~~~| \hilight{orange}{otherwise}~= stream s} \\
\texttt{~~~~\hilight{orange}{where}~foot = \hilight{orange}{not}~\$ lastFoot s} \\
\texttt{~~~~~~~~~~\hilight{commentblue}{-{}- record whether we stepped on a matching or opposite L/R arrow here}} \\
\texttt{~~~~~~~~~~addStep ft \hilight{brickred}{L}~steps = ft:steps} \\
\texttt{~~~~~~~~~~addStep ft \hilight{brickred}{R}~steps = (\hilight{orange}{not}~ft):steps} \\
\texttt{~~~~~~~~~~addStep ft \_ steps = steps \hilight{commentblue}{-{}- U/D don't help to determine L/R footing.}} \\
\texttt{~~~~~~~~~~\hilight{commentblue}{-{}- add the (matching? xover?) step to the list corresponding to this foot.}} \\
\texttt{~~~~~~~~~~newLs s = \hilight{orange}{if}~foot \hilight{orange}{then}~addStep foot step \$ stepsL s \hilight{orange}{else}~stepsL s} \\
\texttt{~~~~~~~~~~newRs s = \hilight{orange}{if}~foot \hilight{orange}{then}~stepsR s \hilight{orange}{else}~addStep foot step \$ stepsR s} \\
\texttt{~~~~~~~~~~stream s = s \{ steps = steps s + \hilight{brickred}{1}, lastStep = \hilight{brickred}{Just}~step,} \\
\texttt{~~~~~~~~~~~~~~~~~~~~~~~~ lastFoot = foot, stepsL = newLs s, stepsR = newRs s \} } \\
\texttt{} \\
\texttt{\hilight{pink}{analyze}~:: [\hilight{olivegreen}{Step}] -> \hilight{olivegreen}{AnalysisState}} \\
	\texttt{\hilight{pink}{analyze}~= \hilight{pink}{commitStream}~. \hilight{orange}{foldl}~\hilight{pink}{analyzeStep}~(\hilight{brickred}{S 0 0 0 0}~\hilight{brickred}{Nothing}~\hilight{brickred}{False}~\hilight{brickred}{False}~\hilight{brickred}{False}~\hilight{brickred}{[] []}) } \\
\end{tabular}
\end{center}
\caption{it's pseudocode lol}
\label{fig:pseudocode-lol}
\end{figure*}


\section{Existing Chart Analysis}


\section{Conclusion}

Please accept our paper.
We worked hard on it.

%%%%%%%%%%%%%%%%%%%%%%%%%%%%%%%%%%%%%%%%%%%%%%%%%%%%%%%%%%%%%%%%%%%%%%%%%%%%%%%%

\bibliographystyle{abbrvnat}
\bibliography{citations}

\end{document}
