%-----------------------------------------------------------------------------
%
%               Template for sigplanconf LaTeX Class
%
% Name:         sigplanconf-template.tex
%
% Purpose:      A template for sigplanconf.cls, which is a LaTeX 2e class
%               file for SIGPLAN conference proceedings.
%
% Guide:        Refer to "Author's Guide to the ACM SIGPLAN Class,"
%               sigplanconf-guide.pdf
%
% Author:       Paul C. Anagnostopoulos
%               Windfall Software
%               978 371-2316
%               paul@windfall.com
%
% Created:      15 February 2005
%
%-----------------------------------------------------------------------------


%\documentclass[preprint]{sigplanconf}
\documentclass[onecolumn]{sigplanconf}

% The following \documentclass options may be useful:
%
% 10pt          To set in 10-point type instead of 9-point.
% 11pt          To set in 11-point type instead of 9-point.
% authoryear    To obtain author/year citation style instead of numeric.

\usepackage{yfonts}
\usepackage{amsmath}
\usepackage{amsthm}
\usepackage{amssymb}
%\usepackage{mathpartir}
\usepackage{hyperref}
\usepackage{url}
\usepackage{graphics}
\usepackage{graphicx}
\usepackage{wasysym}
\usepackage{harmony}
\usepackage{marvosym}
\usepackage{multirow}
\usepackage[usenames,dvipsnames]{xcolor}
\usepackage[utopia]{mathdesign}
\usepackage{natbib}
\usepackage[mathcal]{euscript}
\usepackage[linesnumbered,ruled]{algorithm2e}

\renewcommand{\UrlBreaks}{\do\/\do\a\do\b\do\c\do\d\do\e\do\f\do\g\do\h\do\i\do\j\do\k\do\l\do\m\do\n\do\o\do\p\do\q\do\r\do\s\do\t\do\u\do\v\do\w\do\x\do\y\do\z}

% ____________________________________________________________
% Listings Package Configuration
% \usepackage[scaled]{beramono}

%\renewcommand*\ttdefault{txtt}
\usepackage[T1]{fontenc}

% This Deep Tex Voodoo is from
%   <http://www.latex-community.org/forum/viewtopic.php?f=5&t=2072>
% It's purpose is to make \lstinline normal size, without affecting
% \lstinputlisting.  It seems to work but I have no idea how or why,
% and I rather hope never to learn.
%\makeatletter
%\lst@AddToHook{TextStyle}{\let\lst@basicstyle\ttfamily\normalsize}
%\makeatother

\begin{document}

\conferenceinfo{SIGBOVIK '18}{Pittsburgh, PA, USA}
\copyrightyear{2018}
\copyrightdata{}

\titlebanner{banner above paper title}        % These are ignored unless
\preprintfooter{short description of paper}   % 'preprint' option specified.

\title{
Retraction of \\
a boring follow-up paper to \\
``Which ITG Stepcharts are Turniest?'' \\
titled, ``Which ITG Stepcharts are Crossoveriest and/or Footswitchiest?'' \\
%That `Boring' Stuff Was Part of the Title, BTW. \\
%So was that. And that, and this too. \\
%You got it all, right? \\
%Or Just, ``More Boring Crap about ITG'', for Short. \\
%Oh, That Was Also Part of the Title.
}
% \subtitle{\em The Randomly-Scoped Lambda Calculus}
% \subtitle{Subtitle Text, if any}

\authorinfo{Ben Blum}{}{bblum@alumni.cmu.edu}

\maketitle

In my 2017 paper, a boring follow-up paper to {\em Which ITG Stepcharts are Turniest?} titled,
{\em Which ITG Stepcharts are Crossoveriest and/or Footswitchiest?} \cite{crossoveriness},
I wrote of maximum $\mathcal{T}$, XO\%, FS\%, and JK\% values as follows:

\begin{center}
\begin{tabular}{p{0.87\textwidth}}
	A chart could conceivably end right before such a step,
	sneaking through some small $\epsilon$ extra turniness \cite{epsilon}
	(similar to the case of 270s in \cite{turniness}),
	\\
	\\
	{[\dots]}
	\\
	\\
	By the way, the theoretical maxima for XO\%, FS\%, and JK\%
	are 50-$\epsilon$, 100-$\epsilon$, and 100-$\epsilon$, respectively \cite{epsilon}.
\end{tabular}
\end{center}

\noindent
However,
in the experimental results, {\em Tachyon Epsilon} \cite{tachyon-epsilon}
placed among the lowest-ranking stepchart packs in every category,
yet I neglected to properly cite Dr. VII's landmark paper from 2014 at that time.
Though we may never know know what, if anything, is Epsilon?
we now know at least what it is not: crossovery and/or footswitchy.

In conclusion, please reject my paper.
I messed up on it.

%%%%%%%%%%%%%%%%%%%%%%%%%%%%%%%%%%%%%%%%%%%%%%%%%%%%%%%%%%%%%%%%%%%%%%%%%%%%%%%%

\bibliographystyle{abbrvnat}
\bibliography{citations}

\end{document}
